% !TEX root = main.tex
\section{Hintergrund}
\label{hintergrund}
In Vorbereitung auf die Antragstellung bei der Deutschen Forschungsgemeinschaft (DFG) zur Förderung eines Open-Access-Publikationsfonds wurde an der Technischen Universität Berlin (TU Berlin) ein Verfahren für die Analyse des Publikationsaufkommen und den Anteil an Open Access (OA) entwickelt. Dieses Verfahren wurde weiterentwickelt für die Analyse des OA-Anteils wissenschaftlicher Zeitschriftenartikel der Angehörigen von Berliner Bildungs- und Forschungseinrichtungen.

Für die DFG-Antragsstellung benötigt wurden Aussagen über das Aufkommen von Zeitschriftenartikeln von TU-Angehörigen für die  Jahre 2014 und 2015, insbesondere über den Anteil von Artikeln in Open-Access-Zeitschriften (Anteil OA Gold). Für die Analyse des Berliner Publikationsaufkommens von neun verschiedenen Einrichtungen wurden die Jahre 2013 bis 2015 analysiert. Für die Analyse der Berliner Artikeldaten für das Jahr 2016 wurde das Skript vollständig überarbeitet (insbesondere hinsichtlich Datenimport) sowie funktional erweitert (Anteil OA Grün).

Für die Analyse wurde auf Daten aus zehn bzw. sechzehn externen Literatur- und Zitations\-datenbanken zurückgegriffen.
Die gewonnenen Daten zu den Dokumententypen \texttt{Article} bzw. \texttt{Review} wurden normalisiert, aggregiert und auf Dubletten geprüft. Um Artikel aus Open-Access-Zeitschriften zu identifizieren, wurden Daten des Directory of Open Access Journals (DOAJ)\footnote{Directory of Open Access Journals (DOAJ) s.~\url{http://doaj.org}} genutzt. In den verbleibenden Daten von OA-Artikeln wurden im Folgenden diejenigen Artikel identifiziert, für die Angehörige der untersuchten Einrichtungen als Erst- oder Korrespondenz-autoren angegeben sind. Um OA-Artikel in Closed-Access-Zeitschriften sowie den Anteil an Artikeln, die über den Grünen Weg Open Access verfügbar gemacht werden, zu identifizieren, wird die Schnittstelle von Unpaywall\footnote{API von Unpaywall s.~\url{https://unpaywall.org/api/v2}} abgefragt.

Bei der Datenerhebung wurden folgende Datenbanken berücksichtigt: Web of Science Core Collection, SciFinder (CAPlus), PubMed, TEMA, Inspec, IEEE Xplore, ProQuest Social Sciences, Business Source Complete, GeoRef, CAB Abstracts, CINAHL, Academic Search Premier, Embase, LISA, Scopus, Sport Discus.

Zur Unterstützung der Evaluation wurde ein Python-Skript entwickelt, dessen Funktionsweise ab S.~\pageref{funktionsweise} beschrieben wird. Ab S.~\pageref{analysis} wird erläutert, welche Schritte für eine eigene Analyse mithilfe dieses Skripts durchzuführen sind.

Ob mit dem hier beschriebenen Verfahren alle OA-Artikel einer bestimmten Einrichtung identifiziert werden können, bleibt offen. Folgende Faktoren stellen potentielle Fehlerquellen dar:
\begin{itemize}
\item Artikel in Open-Access-Zeitschriften, die nicht in einer der geprüften Datenbanken indexiert sind, werden nicht berücksichtigt.
\item Die Artikel werden über externe Datenbanken ermittelt; Voraussetzung für die Identifizierung ist das Erfassen der Affiliation in diesen Datenbanken. Es werden hier zwar Affiliationen für alle Autorinnen und Autoren erfasst -- allerdings pro Autor bzw. Autorin meist nur eine Affiliation. Bei Mehrfachaffiliationen wird i.\,d.\,R. nur eine Institution in der Datenbank erfasst.
\item Open-Access-Zeitschriften werden mithilfe des DOAJ identifiziert. Ist eine Zeitschrift nicht im DOAJ erfasst, werden Open-Access-Artikel nicht als solche erkannt. Es ist ebenso möglich, dass die Zeitschrift zum Zeitpunkt der Publikation des Artikels noch als Open-Access-Zeitschrift im DOAJ gelistet wurde, zum Zeitpunkt der Analyse aber aus dem DOAJ entfernt wurde.
\item Während in einigen Datenbanken, z.B. Web of Science (WoS) oder Scopus, die Korrespondenzautorin bzw. der Korrespondenzautor gesondert ausgewiesen wird\footnote{WOS: \textit{reprint author}, Scopus: RIS-Feld N1}, werden in anderen Datenbanken lediglich Affiliationen erfasst. Ein eindeutiger (automatisch gestützter) Rückschluss auf die Korrespondenzautorschaft ist für diese Daten nicht möglich. Es werden für diese Datenbanken daher lediglich die Institutionsangaben zu Erstautorinnen bzw. Erstautoren evaluiert. Nicht in allen Disziplinen aber sind Korrespondenz- und Erstautorin identisch. Open-Access-Artikel, für die Angehörige einer Einrichtung Korrespondenz- nicht aber Erstautoren sind, bleiben somit unentdeckt.

Eine Analyse der Daten des Jahres 2016 hat ergeben, dass in 14 \% der Fälle keine explizite Angabe zur Korrespondenzautorschaft vorhanden war, woraufhin in erster Approximation der Erstautor bzw. die Erstautorin herangezogen wurde. Die Auswertung der Daten aus WoS zeigt, dass dies in ca. 64 \% der Fälle korrekt ist. Insgesamt ergibt sich also eine daraus resultierende Fehlerquote von ca. 5 \%.
\end{itemize}
