% !TEX root = main.tex
\clearpage
\begin{appendices}
\label{appendix}
%\subsection*{Übersicht Skriptdateien}
\begin{table}[h]
\caption{Übersicht der einzelnen Skriptdateien}
    \begin{tabularx}{1\textwidth}{p{5cm}p{9cm}}
    \toprule
     Skriptdatei & Erläuterung \\
     \midrule
main.py & Hauptskript\\
RIS-fields.csv & Übersicht RIS-Felder der verschiedenen Datenbanken\\
%
    \bottomrule
    \end{tabularx}
%\caption{Übersicht der einzelnen Skriptdateien}
\end{table}

%\subsection*{Übersicht Eingabedateien}
\begin{table}[h]
\caption{Übersicht der vom Skript einzulesenden Dateien (Unterordner \texttt{input-files})}
    \begin{tabularx}{1\textwidth}{p{5cm}p{9cm}}
    \toprule
     Eingabedatei & Erläuterung \\
     \midrule
     docsChecked.txt & Publikationen, für die Affiliation der Erst- bzw. Korrespondenzautorschaft manuell geprüft wurde \\
    doaj.txt & DOAJ-Metadaten \\
    \midrule
    bsc20xx.txt & Artikeldaten Business Source Complete \\
    cab20xx.txt & Artikeldaten CAB Abstracts \\
    cinahl20xx.txt & Artikeldaten CINHAL \\
    ebsco20xx.txt & Artikeldaten Academic Search Premier (EBSCO) \\
    embase20xx.txt & Artikeldaten Embase \\
    gf20xx.txt & Artikeldaten GeoRef \\
    ieee20xx.txt & Artikeldaten IEEE Xplore \\
    inspec20xx.txt & Artikeldaten InSpec \\
    lisa20xx.txt & Artikeldaten LISA \\
    pq20xx.txt & Artikeldaten ProQuest Social Sciences \\
    pubmed20xx.txt & Artikeldaten PubMed \\
    scopus20xx.txt & Artikeldaten Scopus \\
    sd20xx.txt & Artikeldaten Sport Discus \\
    sf20xx.txt & Artikeldaten SciFinder \\
    tema20xx.txt & Artikeldaten TEMA \\
    wos20xx.txt & Artikeldaten Web of Science \\
    \bottomrule
    \end{tabularx}
%\caption{Übersicht der vom Skript einzulesenden Dateien}
\end{table}

%\subsection*{Übersicht Skriptvariablen}
\begin{table}[h]
\caption{Übersicht der wichtigsten im Skript verwendeten Variablen}
    \begin{tabularx}{1\textwidth}{p{5cm}p{9cm}}
    \toprule
     Variable im Skript & Erläuterung \\
     \midrule
checkToDo & Skriptfunktionalität (de-)aktivieren: manuelle Prüfung Erst-/Korrespondenzautorschaft\\
contactCR & Skriptfunktionalität (de-)aktivieren: Abfragen der API von Crossref \\
contactOaDOI & Skriptfunktionalität (de-)aktivieren: Abfragen der API von Unpaywall \\
doAnalysis & Skriptfunktionalität (de-)aktivieren: Datenauswertung (Statistik, Diagramme)\\
doReadIn & Skriptfunktionalität (de-)aktivieren: Einlesen Artikeldaten aus Datenbankrecherchen\\
finalList & Liste aller gefundenen Artikel\\
yearMin & Start des Untersuchungszeitraums; es ist das Jahr mit vier Stellen anzugeben (relevant für korrekte Erkennung der PubMed-Daten)\\
yearMax & Ende des Untersuchungszeitraums, es ist das Jahr  mit vier Stellen anzugeben (relevant für korrekte Erkennung der PubMed-Daten)\\
    \bottomrule
    \end{tabularx}
\end{table}

%\subsection*{Übersicht Ausgabedateien}
\begin{table}[h]
\caption{Übersicht der vom Skript ausgegebenen Dateien (Unterordner \texttt{output-files})}
    \begin{tabularx}{1\textwidth}{p{5cm}p{9cm}}
    \toprule
     Ausgabedatei & Erläuterung \\
     \midrule
    allPubs.txt & Liste aller gefundenen Artikel \\
    docsCheckedCantFind.txt & Hilfsdatei, wird nur dann erstellt, wenn beim Einlesen der Datei \texttt{docsChecked.txt} ein Fehler auftritt. (z.\,B. aufgrund eines Copy/Paste-Fehlers bei der Erstellung der txt-Datei) \\
    %
    docsToBeChecked.txt & Liste der Publikationen, für die Affiliation der Erst- bzw. Korrespondenzautorschaft manuell zu prüfen ist \\
    finalList & interne Arbeitsdatei aller Artikeldaten \\
    oaDOI-response.txt & Werte aus Unpaywall-Schnittstelle  \\
    %
    statistics\_OA.txt & Statistik der analysierten Artikel \\
    statistics\_goldPublishers.txt & Häufigkeitsverteilung der Gold-Artikel (OA-Artikel in OA-Zeitschriften) auf Verlage \\
	DOIs-oaDOI-error.txt & Liste der DOIs, für die Unpaywall eine Fehlermeldung zurückgegeben hat \\
	\bottomrule
    \end{tabularx}
\end{table}

\begin{table}[h] \small
\caption{Angaben in allPubs.txt} \label{tab:art-data}
    \begin{tabularx}{1\textwidth}{p{3cm}p{2.5cm}p{8cm}} 
    \toprule
     Feld & Quelle(n) & Erläuterung \\
     \midrule
	authors & Datenbanken & Angaben gekürzt auf max. 2500 Zeichen \\
 	title & Datenbanken & \\
	OA-Status & DOAJ, Unpaywall & Angaben, über welchen Weg Artikel OA verfügbar ist (\texttt{None}, \texttt{gold}, \texttt{hybrid} oder \texttt{green}) \\
	DOI & Datenbanken & \\
 	journal & Datenbanken &  \\
	ISSN & Datenbanken, Crossref & ISSN der Print- oder Onlineausgabe (wahrscheinlich Print)\\
	eISSN & Datenbanken, Crossref & ISSN der Print- oder Onlineausgabe (wahrscheinlich Online)\\
	publisher & Datenbanken, DOAJ, \mbox{Unpaywall} & für OA-Artikel werden ursprüngliche Angaben zwecks Vereinheitlichung überschrieben\\
	year & Datenbanken & PubMed indiziert verschiedene Daten: PubMed-Suche deckt alle Datumsfelder ab, während das Python-Skript nur ein Datumsfeld auswertet (\texttt{DP = Date of Publication})\\
	affiliations & Datenbanken & Affiliationen der Autorinnen und Autoren\\
	all identified name variants & Skript & Institutionskürzel für alle vorhandenen Affiliationen, auf Basis der im Skript angesetzten Institutionsnamen zugeordnet wurden \\
	corresponding author & Datenbanken & Affilation der Korrespondenzautorin bzw. des Korrespondenzautors\\
	found name variant & Skript, ggf. manuell & Institutionskürzel für Affiliation der  Korrespondenzautorschaft, das auf Basis der im Skript angesetzten Institutionsnamen zugeordnet wurde\\
	e-mail & Datenbanken & \\
	subject & Datenbanken & Angaben aus Web of Science, SciFinder \\
	DOAJ subject & DOAJ & im DOAJ werden Zeitschriften nach der Library of Congress Classification klassifiziert\\
	funding & Datenbanken & Angaben aus Web of Science \\
	license & DOAJ, Unpaywall & Standardlizenz für Journal laut DOAJ bzw. Lizenzangabe für Artikel aus Unpaywall \\
	databaseID & Skript & \\
	notes & Skript & verschiedene Indikatoren für Datenquellen: \newline \texttt{Checked by hand.} = Affiliation für Korrespondenzautorschaft manuell ermittelt; \newline \texttt{Identified via DOAJ} = OA-Status ermittelt mithilfe von DOAJ; \newline \texttt{Identified via Unpaywall} = OA-Status ermittelt mithilfe von Unpaywall\\
    oaDOI[is\_oa] & Unpaywall & \texttt{True} = Artikel ist OA verfügbar \\
    oaDOI[journal\_is\_oa] & Unpaywall & \texttt{True} = Journal ist OA-Zeitschrift \\
    oaDOI[host\_type] & Unpaywall & primäre OA-Version verfügbar über Verlag (\texttt{publisher}) oder gesichertes Repositorium (\texttt{repository})\\
    oaDOI[license] & Unpaywall & \\
    APC Amount & DOAJ & \\
    APC Currency & DOAJ & \\
	\bottomrule
    \end{tabularx}
\end{table}

\end{appendices}